\newcommand{\printOnMatch}[3]{ % var, expected value, to print
  % \ifthenelse{\equal{\progPos}{II}}{thing is `whatever'.}{thing is something else.}
  % \ifthenelse{\equal{#1}{#2}}{#3}{\phantom{#3}}
  % \ifthenelse{\equal{#1}{#2}}{#3}{\textcolor{red}{#3}}
  \ifthenelse{\equal{#1}{#2}}{\textcolor{black}{#3}}{\textcolor{white}{#3}}
}

\newcommand{\printITE}[4]{ % var, expected value, to print if true, to print if false
  \ifthenelse{\equal{#1}{#2}}{#3}{#4}
}


\newcommand{\printOnMatchColor}[4]{ %  % var, expected value, to print, color
  \ifthenelse{\equal{#1}{#2}}{\textcolor{#4}{#3}}{\textcolor{white}{#3}}
}

\newcommand{\progPosI}{O} \newcommand{\progPosII}{O}
\newcommand{\activeColor}{blue} \newcommand{\passiveColor}{gray}
\newcommand{\tIcolor}{\activeColor} \newcommand{\tIIcolor}{\passiveColor}
\newcommand{\ppI}[1]{%
  \printOnMatchColor{\progPosI}{#1}{\blacktriangleright}{\tIcolor}
}
\newcommand{\ppII}[1]{%
  \printOnMatchColor{\progPosII}{#1}{\blacktriangleright}{\tIIcolor}
}
\newcommand{\selectI}{
  \renewcommand{\tIcolor}{\activeColor} \renewcommand{\tIIcolor}{\passiveColor}
}
\newcommand{\selectII}{
  \renewcommand{\tIcolor}{\passiveColor} \renewcommand{\tIIcolor}{\activeColor}
}
\newcommand{\unselect}{
  \renewcommand{\tIcolor}{\passiveColor} \renewcommand{\tIIcolor}{\passiveColor}
}

\newcommand{\upd}[2]{\renewcommand{#1}{#2}}
\newcommand{\showConcrete}{false}
\newcommand{\repeatCell}[1]{\only<4->{$#1 \kw{repeat}\ \{\}$}}
\newcommand{\untilCell}[1]{\only<1-3>{$#1 lock();$}\only<4->{$#1 \kw{until}\ CAS(l, 0, 1);$}}
\newcommand{\writeCell}[1]{\only<1-3>{$#1 unlock();$}\only<4->{$#1 \writeInst{l}{0};$}}
% \newcommand{\topStyle}{l}
% \newcommand{\specialBorder}{0pt}
\newcommand{\sbc}{white}
\newcommand{\spinlockClientIIExpandedIter}{
  % \only<4->{\upd{\specialBorder}{0.5pt}}
  \only<4->{\upd{\sbc}{black}}
\begin{figure}[h]
  \centering
  
    % \begin{tabular}{l | l}
    % \arrayrulecolor{white}
  \begin{tabular}{p{3cm} | p{3cm}}
  % \begin{tabular}{p{3cm} !{\textcolor{\sbc}{\vrule}} p{3cm}}  
      % \multicolumn{2}{c}{$\writeInst{l}{0}{}$} \\
    \multicolumn{2}{c}{\onslide<4->{$\writeInst{l}{0}{}$}} \\
      % \hline
      \noalign{{\color{\sbc}\hrule height 0.5pt}}
      % \noalign{\hrule}
      % \onslide<4->{\arrayrulecolor{black}}
      % \repeatCell{\ppI{I}} & \repeatCell{\ppII{I}} \\      \
      % \only<1-3>{\multicolumn{2}{c}{foobar}}\only<4->{\repeatCell{\ppI{I}} & \repeatCell{\ppII{I}}} \\
      % \only<1-3>{\multicolumn{2}{c}{foobar}}\only<4->{fas & fasf} \\
      % \multicolumn{1}{\topStyle}{foo} & \multicolumn{1}{r|}{bar} \\
      \multicolumn{1}{l}{\repeatCell{\ppI{I}}} & \multicolumn{1}{!{\textcolor{\sbc}{\hspace{-0.015cm}\vrule}}l}{\repeatCell{\ppII{I}}} \\
      
      \untilCell{\ppI{II}} & \untilCell{\ppII{II}} \\
      $\ppI{III} \comment{critical section}$ & $\ppII{III} \comment{critical section}$ \\
      \writeCell{\ppI{IV}} & \writeCell{\ppII{IV}} \\
      % \multicolumn{1}{l}{\untilCell{\ppI{II}}} & \multicolumn{1}{!{\textcolor{\sbc}{\vrule}}l}{\untilCell{\ppII{II}}} \\

    \end{tabular}
  \end{figure}
}

%%% Local Variables:
%%% mode: latex
%%% TeX-master: "oopsla"
%%% End:
