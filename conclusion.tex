\begin{frame}{Further work}
  \begin{columns}
    \begin{column}{0.4\linewidth}
      \begin{center}
      \includegraphics[width=0.6\linewidth]{cpp11.png}

      % \vspace{0.4cm}
      Richer memory accesses
    \end{center}
    \end{column}
    \begin{column}{0.4\linewidth}

      \vspace{0.5cm}
      
      \begin{minipage}[c]{0.45\textwidth}
        \includegraphics[width=1.0\linewidth]{arm.png}
      \end{minipage}
      \hfill
      \begin{minipage}[c]{0.45\textwidth}
        \includegraphics[width=1.0\linewidth]{power.png}
      \end{minipage}
            
      \begin{center}
      {\large \sout{$acyclic(\lPO \cup \lRF)$}}
    \end{center}
    \end{column}
    
  \end{columns}
  
\end{frame}

\begin{frame}{Takeaway}
% \begin{frame}
  
  % {\large Summary:}
  % \begin{itemize}
  % \item We describe fairness for operational memory models
  % \item And provide an equivalent uniform declarative definition
  % \item Which is used to prove lock algorithms termination.
  % \end{itemize}
  
  % \begin{columns}
  %   \begin{column}{0.5\linewidth}      
  %     \scalebox{0.7}{\fairTrace}
  %   \end{column}    
  %   \begin{column}{0.4\linewidth}
  %     \renewcommand{\hof}{2}
  %     \renewcommand{\vof}{1}
  %     \scalebox{0.8}{
  %     \begin{tikzpicture}[xscale=2, yscale=0.8]
  %       \spinlockContraGraphEventsI
  %       \spinlockContraGraphRelationsI
  %       \spinlockContraGraphEventsII
  %       \spinlockContraGraphRelationsII
  %       \spinlockContraGraphContra
  %     \end{tikzpicture}
  %     }
  %   \end{column}
  % \end{columns}

  \begin{center}
  \scalebox{0.8}{\fairTrace}

  \vspace{0.7cm}  
  \renewcommand{\hof}{2}
  \renewcommand{\vof}{1}
  \scalebox{0.8}{
    \begin{tikzpicture}[xscale=2, yscale=0.8]
      \spinlockContraGraphEventsI
      \spinlockContraGraphRelationsI
      \spinlockContraGraphEventsII
      \spinlockContraGraphRelationsII
      \spinlockContraGraphContra
    \end{tikzpicture}
  }
\end{center}
  % \todo{{\large More in the paper: ?}}

\end{frame}
%%% Local Variables:
%%% mode: latex
%%% TeX-master: "oopsla"
%%% End:
