\begin{frame}{Further work}
  % \begin{columns}
  %   \begin{column}{0.4\linewidth}
  %     \begin{center}
  %       % \includegraphics[width=0.6\linewidth]{cpp11.png}
  %       \includegraphics[width=0.6\linewidth]{cpp.jpg}

  %     % \vspace{0.4cm}
  %     More memory access types
  %   \end{center}
  %   \end{column}
  %   \begin{column}{0.4\linewidth}

  %     \vspace{0.5cm}
      
  %     \begin{minipage}[c]{0.45\textwidth}
  %       \includegraphics[width=1.0\linewidth]{arm.png}
  %     \end{minipage}
  %     \hfill
  %     \begin{minipage}[c]{0.45\textwidth}
  %       \includegraphics[width=1.0\linewidth]{power.png}
  %     \end{minipage}
            
  %     \begin{center}
  %     {\large \sout{$acyclic(\lPO \cup \lRF)$}}
  %   \end{center}
  %   \end{column}
    
  % \end{columns}

  \begin{center}
    \hfill
    \includegraphics[width=0.15\linewidth]{cpp.png} \hfill
    \includegraphics[width=0.3\linewidth]{arm.png} \hfill
    \raisebox{0.5cm}{\includegraphics[width=0.3\linewidth]{power.png}} \hfill
  \end{center}

  \vspace{0.5cm}
  \renewcommand{\ULthickness}{0.75pt}
  \begin{center}
    {\large More memory access types,\ \   \sout{$acyclic(\lPO \cup \lRF)$}}
\end{center}

  %   \begin{center}
  %   \begin{minipage}{0.5\linewidth}
  %   \begin{itemize}
  %   \item More memory access types
  %   \item {\large \sout{$acyclic(\lPO \cup \lRF)$}}
  %   \end{itemize}
  %   \end{minipage}
  % \end{center}
\end{frame}

\begin{frame}{Takeaway}
% \begin{frame}
  
  % {\large Summary:}
  % \begin{itemize}
  % \item We describe fairness for operational memory models
  % \item And provide an equivalent uniform declarative definition
  % \item Which is used to prove lock algorithms termination.
  % \end{itemize}

  \vspace{-0.5cm}
  
  \begin{columns}
    \begin{column}{0.4\linewidth}
      \scalebox{0.9}{\fairTrace}
    \end{column}    
    \begin{column}{0.4\linewidth}
      \renewcommand{\hof}{2}
      \renewcommand{\vof}{1}
      \scalebox{0.9}{
      \begin{tikzpicture}[xscale=2, yscale=0.9]
        \spinlockContraGraphEventsI
        \spinlockContraGraphRelationsI
        \spinlockContraGraphEventsII
        \spinlockContraGraphRelationsII
        \spinlockContraGraphContra
      \end{tikzpicture}
      }
    \end{column}    
  \end{columns}

  \vspace{0.5cm}
  \begin{center}
      % Notion of memory fairness allows to prove termination under weak memory
      % By requiring memory fairness, we're now able to prove termination under weak memory
      % By introducing memory fairness, we're able to prove termination under weak memory
    Memory fairness requirement allows to prove termination under weak memory
  \end{center}

%   \begin{center}
%   \scalebox{0.8}{\fairTrace}

%   \vspace{0.7cm}  
%   \renewcommand{\hof}{2}
%   \renewcommand{\vof}{1}
%   \scalebox{0.8}{
%     \begin{tikzpicture}[xscale=2, yscale=0.8]
%       \spinlockContraGraphEventsI
%       \spinlockContraGraphRelationsI
%       \spinlockContraGraphEventsII
%       \spinlockContraGraphRelationsII
%       \spinlockContraGraphContra
%     \end{tikzpicture}
%   }
% \end{center}
%   % \todo{{\large More in the paper: ?}}

\end{frame}
%%% Local Variables:
%%% mode: latex
%%% TeX-master: "oopsla"
%%% End:
