% takes a list of lists of the form: {foo=1, bla},{gar=-} and then defines empty macros for each name
\newcommand*{\clearNamesListofLists}[1]{%
    \forcsvlist{    \clearNamesList}{#1}%
}
\newcommand*{\clearNamesList}[1]{%
    \forcsvlist{    \clearName}{#1}%
}
\newcommand*{\clearName}[1]{%
    \getItemName{#1}{cmdname}%
    \csdef{\cmdname}{}% 
  }


\newcounter{mycounter}

% #1 is of the form <name>[=-|n], e.g. foo=- or foo=10 or foo
% Stores in #2 the overlay specification for <name> s.t. it can be appended to the content of <name>
\newcommand*{\getNewOverlayContent}[2]{%
    \getItemSpec{#1}{itemSpec}%
    \IfStrEq{\itemSpec}{-}{%
        \csedef{#2}{\arabic{beamerpauses}-}%
    }{%
        \IfStrEq{\itemSpec}{}{%
            \csedef{#2}{\arabic{beamerpauses}}%
        }{%
            \IfInteger{\itemSpec}{%
%               \mycounter=\
                \setcounter{mycounter}{\arabic{beamerpauses}}%
                \addtocounter{mycounter}{\itemSpec}%
                \addtocounter{mycounter}{-1}%
                \csedef{#2}{\arabic{beamerpauses}-\arabic{mycounter}}%
            }{%
                \PackageError{setorder}{Argument has illegal format}{Argument was #1}%
            }%
        }%
    }%
%   input: #1, itemspec:\itemSpec, beamervalue: \arabic{beamerpauses}, content: \csuse{#2} \\
}
% #1 is of the form 'foo=1' or 'foo=-' or 'foo'. 
% #2 Is the name of the macro which should hold the result
% This macro stores the part infront '=' (the name) in #2.
\newcommand*{\getItemName}[2]{% 
    \IfSubStr{#1}{=}{%
        \StrBefore{#1}{=}[\tmp]%
        \csdef{#2}{\tmp}%
    }{%
        \csdef{#2}{#1}% 
    }%
}
% #1 is of the form 'foo=1' or 'foo=-' or 'foo'. 
% #2 Is the name of the macro which should hold the result
% This macro stores the part behind '=' (the overlay spec) in #2. The stored part is empty iff there is no '=' in #1
\newcommand*{\getItemSpec}[2]{%
    \StrBehind{#1}{=}[\tmp]%
    \csdef{#2}{\tmp}%
}
% #2 is the name where content should be appended. 
% It has been ensured previously that #2 is a defined macro
% #1 is the content to append
% Depending on whether #2 is empty or not a (,) is added 
% before appending #1
\newcommand*{\appendToOverlaySpecification}[2]{%
    \IfStrEq{\csexpandonce{#2}}{}{%
        % #1 i.e. <name> is empty
        \cseappto{#2}{\csname#1\endcsname}%
    }{%
        \cseappto{#2}{,\csname#1\endcsname}%
    }%
}

\newcommand*{\setorderItem}[1]{%
    \getNewOverlayContent{#1}{overlaycontent}%
    \getItemName{#1}{cmdname}%
    \appendToOverlaySpecification{overlaycontent}{\cmdname}%
}

\newcommand*{\setorderList}[1]{%
    \forcsvlist{\setorderItem}{#1}%
    \stepcounter{beamerpauses}%
}
\newcommand*{\setorder}[1]{%    
    \clearNamesListofLists{#1}%
    \forcsvlist{\setorderList}{#1}%
    %% \createBef{#1}%
}
  
%%% Local Variables:
%%% mode: latex
%%% TeX-master: "oopsla"
%%% End:
